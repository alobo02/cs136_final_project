\documentclass[12pt]{article}
\usepackage{fullpage} 
\usepackage{microtype}      % microtypography
\usepackage{array}
\usepackage{amsmath,amssymb,amsfonts}
\usepackage{amsthm}

%% Header
\usepackage{fancyhdr}
\fancyhf{}
\fancyhead[C]{CS 136 - 2022s - Checkpoint1 Submission}
\fancyfoot[C]{\thepage} % page number
\renewcommand\headrulewidth{0pt}
\pagestyle{fancy}

\usepackage[headsep=0.5cm,headheight=2cm]{geometry}

%% Hyperlinks always blue, no weird boxes
\usepackage[hyphens]{url}
\usepackage[colorlinks=true,allcolors=black,pdfborder={0 0 0}]{hyperref}

%%% Doc layout
\usepackage{parskip}
\usepackage{times}

%%% Write out problem statements in blue, solutions in black
\usepackage{color}
\newcommand{\officialdirections}[1]{{\color{blue} #1}}

%%% Avoid automatic section numbers (we'll provide our own)
\setcounter{secnumdepth}{0}

\begin{document}
~~\\ %% add vert space

{\Large{\bf Student Names: TODO}}


{\Large{\bf Collaboration Statement:}}

Turning in this assignment indicates you have abided by the course Collaboration Policy:

\url{www.cs.tufts.edu/comp/136/2022s/index.html#collaboration-policy}

Total hours spent: TODO

We consulted the following resources:
\begin{itemize}
\item TODO
\item TODO
\item $\ldots$	
\end{itemize}

These are the official instructions for checkpoint 1.  You can find instructions on how to submit at \url{www.cs.tufts.edu/comp/136/2022s/checkpoint1.html}

\textbf{Please consult the full project description at \url{https://www.cs.tufts.edu/comp/136/2022s/project.html} in addition to this document when working on this checkpoint.  It gives details on what we expect when choosing a model and coming up with performance hypotheses.}

\newpage

\section{Exploratory Data Analysis}

\subsection{Describing your analysis}

We followed the following data analysis steps to obtain a comprehensive understanding of our data:

\begin{enumerate}
\item Visualize the marginal distributions of each feature
\item Visualize the marginal distribution of the output class
\item Analyzed the correlation between features
\item Conducted principal component analysis (PCA)
\item Visualized the joint distributions of pairs of features:
\begin{enumerate}
\item highly positively correlated features
\item highly negatively correlated features
\item fairly uncorrelated features
\end{enumerate}
\item Visualized the distribution of correlations between each feature and the output class
\end{enumerate}

\subsubsection{General information about the data:}

This dataset contains measures of distances within different shapes (conformations) of a set of 102 molecules. The study that this data comes from used human experts to judge the smell of each molecule and determine whether it is characterized as "musk" or "non-musk", which makes this a binary classification dataset.
\begin{itemize}
\item There are 6,598 conformations total (number of data points)
\item There are 166 features (not including the molecule and conformation names)
\end{itemize}

\subsubsection{Step 1}

After importing the data into a pandas dataframe and standardizing the data using a MinMaxScaler from sklearn, the  hist method was used to visualize the histograms for each of the features. We also visualized the marginal distribution of the output class which is binary.


\subsection{Analyzing the results}


\section{Model and Learning Method Properties}
  

\subsection{Model}


\subsection{Learning method (inference)}

\section{Performance Hypotheses}

\end{document}